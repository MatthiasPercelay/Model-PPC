\documentclass[a4paper]{article}       % onecolumn (second format)

\usepackage[utf8]{inputenc}
\usepackage{lmodern}
\usepackage{graphicx}
\usepackage{amsmath}
\usepackage{amssymb}
 \usepackage[numbers,sort&compress]{natbib}
% ~a4wide
\usepackage{geometry}
\geometry{
includeheadfoot,
margin=2.54cm
}

\graphicspath{{fig/}}
\DeclareGraphicsExtensions{.pdf,.png,.jpg}

\usepackage{paralist}

\usepackage[pdftex,dvipsnames,usenames]{xcolor}
\usepackage[pdftex,colorlinks=true,urlcolor=ForestGreen,citecolor=Blue,linkcolor=BrickRed]{hyperref}
\usepackage{graphicx}
\usepackage{url}


\usepackage[draft, footnote, marginclue, nomargin,index]{fixme}
\fxusetheme{color}
\definecolor{fxtarget}{rgb}{1,0,0.2}
\fxsetface{target}{\bfseries}

\usepackage[scriptsize]{subfigure}
\subfigtopskip=0pt
\subfigcapskip=0pt
\subfigbottomskip=0pt

\usepackage{array}

\usepackage{booktabs}
\title{Nurse Rostering Problem Description}
\author{M2I}
\date{\today}



\begin{document}
\maketitle
\tableofcontents
\listoffixmes


\abstract{

}


% Double interline
% \baselineskip 20pt plus .3pt minus .1pt


\section{Introduction}



\citep{burke.ea-04} ou mieux \citet{burke.ea-04}

\appendix

\section{Description of the problem (In French)}
The description is written in French for validation by head nurses.

\subsection{Affectations d'un agent}

L'emploi du temps est organisé par jour.
Pour chaque jour, un agent est soit au travail, en repos, ou en congés.
Nous donnons maintenant les différentes affectations d'un agent.

\paragraph{Travail (W)}
Chaque jour travaillé (Work -- W), un agent est affecté à un seul créneau de travail.

Le temps de travail journalier est fixe : 7h42 ;)
Dix minutes de pause sont intégrés au temps de travail, et la durée de présence journalière est donc 7h52.
Ce créneau de travail est le plus souvent dans le service.
\begin{compactitem}
\item Matin (M) : 7h10-14h52
\item Journée (J) : 8h30-16h12
\item Soir (S) : 14h10-21h52
\end{compactitem}
Certains jours, l'agent peut aussi travailler en dehors du service. 
\begin{compactitem}
\item Journée de Formation (FO) : 8h30-16h12
\item Autre service (EX)
\end{compactitem}

\paragraph{Repos et congés (B)}
La plupart des repos et congés (Break -- B) sont fixés à l'avance.
La planification des repos hebdomadaire est le principal degré de liberté.
\begin{compactitem}
\item Repos Hebdomadaire (RH)
\item Repos Aménagé (RA)
\item Repos Compensateur (RC)
\item Réduction du temps de travail (RTT)
\item Congé Annuel (CA)
\item Congé Maladie (CM)
\item Jour Férié (JF)
\end{compactitem}

\paragraph{Valeurs spéciales}
Nous allons définir deux affectations spéciales utiles pour la modélisation du problème.
\begin{compactitem}
\item Not Available (NA) : aucune affectation n'a été décidé. L'algorithme doit prendre une décision.
\item No Decision (ND) : l'affection est ignorée. L'algorithme ne prend aucune décision et ne vérifie pas les contraintes légales pour ce jour.
\end{compactitem}
\subsection{Durée du travail dans la fonction publique hospitalière (FPH)}

Nous allons maintenant préciser les contraintes légales sur la durée de travail et les repos dans la fonction publique hospitalière en interprétant les textes de loi disponibles sur \href{https://www.service-public.fr/particuliers/vosdroits/F573}{service-public.fr}.


\paragraph{Durée hebdomadaire}

\begin{quote}
  La durée de travail effectif, heures supplémentaires comprises, ne peut pas dépasser 48 heures par période de 7 jours glissants (c'est-à-dire de date à date).
\end{quote}
Légalement, un agent ne peut pas travailler plus de 6 jours de suite.
En pratique, un agent ne peut travaille pas plus de 5 jours de suite.
\end{compactitem}

\begin{quote}
  L'agent doit bénéficier d'un repos hebdomadaire de 36 heures consécutives minimum.
\end{quote}
Ici, seules deux séquences de créneaux (S-B-M et S-B-J) ne respectent pas cette contrainte.

\begin{quote}
  Un agent ne peut pas travailler plus de 39 heures hebdomadaires en moyenne (heures supplémentaires non comprises) sur un cycle de travail, ni plus de 44 heures par semaine en cas de cycle irrégulier.
\end{quote}
Un agent travaille 5 jours par semaine en moyenne.


\begin{quote}
  Le nombre de jours de repos est fixé à 4 jours pour 2 semaines, 2 d'entre eux, au moins, devant être consécutifs, dont un dimanche.
\end{quote}
Ce nombre peut varier en fonction de la quotité de travail (temps plein ou partiel).

\paragraph{Cycles de travail}

\begin{quote}
Le travail est organisé selon des périodes de référence dénommées cycles de travail définis par service ou par fonctions.
Le cycle de travail est une période de référence dont la durée se répète à l'identique d'un cycle à l'autre. 
Un cycle ne peut pas être inférieure à la semaine civile (du lundi au dimanche), ni supérieure à 12 semaines. 
Le nombre d'heures de travail effectué au cours des semaines composant le cycle peut être irrégulier. 
Les heures supplémentaires et les repos compensateurs sont décomptés sur la durée totale du cycle. 
\end{quote}

Au vu des contraintes légales, nous fixerons une période de travail multiple de deux semaines (un cycle de deux semaines).

\subsection{Autres contraintes}

Nous précisons maintenant les contraintes tactiques et opérationnelles issues de l'organisation de l'hôpital.

\paragraph{Demande}
Au niveau opérationnel, la demande dans le service est exprimé en nombre d'agents pour chaque créneau (M/J/S) de la période de travail.
\paragraph{Nombre de jours travaillés}
Le nombre de jours de travail pour la période de travail est fixé en fonction de la quotité de travail de l'agent et d'autres facteurs. 

\paragraph{Pré-affectation}
Au niveau tactique, un certain nombres d'affectations ont déjà été fixées et ne peuvent pas être changées.
Certains jours de travail, de repos, ou de congés sont donc imposés aux agents.


\subsection{Objectifs et préférences}

L'objectif est de proposer un emploi du temps satisfaisant la demande, et maximisant la qualité de service et les préférences des agents.
Quand la domaine ne peut pas être satisfaite (le problème est sur-contraint), on cherchera prioritairement à la satisfaire \emph{au mieux}. 
Chaque agent exprime des préférences sur
\begin{inparaenum}[a)]
\item les jours et créneaux travaillés,
\item le nombre de jours de travail consécutifs,
\item les repos de 2 jours consécutifs
\item les repos du week-end.
\end{inparaenum}




\bibliographystyle{unsrtnat}       
\bibliography{bibli-NRP.bib}   % name your BibTeX data base

\end{document}


%%% Local Variables: 
%%% TeX-command-default: "latexmk" 
%%% mode: latex
%%% TeX-master: t
%%% End: 
